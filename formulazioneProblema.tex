\chapter{Impostazione del problema di ricerca}
\label{FormulazioneProblema}
\thispagestyle{empty}

%\begin{quotation}
%{\footnotesize
%\noindent{\emph{``Terence: Rotta a nord con circospezione \\
%Bud: Ehi, gli ordini li do io qui!\\
%Terence: Ok, comante\\
%Bud: Rotta a nord\\
%Terence: Soltanto?\\
%Bud: Con circospezione!''}
%}
%\begin{flushright}
%Chi Trova un Amico Trova un Tesoro
%\end{flushright}
%}
%\end{quotation}
\vspace{0.5cm}

\noindent 
\section{Modello di osservazione}
\subsection{Sfocatura}
Il fenomeno della sfocatura avviene quando un elemento trasparente o semitrasparente si interpone tra la lente della camera e la scena ripresa, causando una perdita nei dettagli della scena ripresa.
Riprendendo \cite{alippi2010detecting}, questo fenomeno pu\`o essere modellato come un operatore di \textit{degradazione} $D$ applicato a un'immagine $y$, considerata priva di errori, i.e.,
\begin{equation}
z=D[y].
\end{equation}
In particolare, all'interno dell'operatore $D$ si pu\`o considerare il contributo dovuto a un operatore di \textit{sfocatura} $B$ (dall'inglese \textit{blur}) e un termine $\eta$ corrispondente al rumore, i.e.,
\begin{equation}
\label{blur_single}
z(x)=D[y](x) = B[y](x) + \eta(x), \qquad x \in X
\end{equation}
dove abbiamo indicato con $x$ le coordinate dei \textit{pixel} dell'immagine. 
Possiamo assumere la sfocatura $B$ come un operatore \textit{lineare} di \textit{convoluzione},
\begin{equation}
B[y](x) = \int_{X}y(x)h(x,s)ds,
\end{equation}
dove $h(x,s)$ rappresenta un filtro \textit{gaussiano} o \textit{uniforme}, il cui risultato consiste nel rendere le differenze di intensit\`a, tra pixel adiacenti, pi\`u morbide (\textit{smooth}).\\
Nel caso pi\`u generale possiamo considerare che la camera acquisisca un sequenza di $N$ osservazioni $\{z_i\}, i = 1, \dots ,N$, quindi la formula \ref{blur_single} si pu\`o riscrivere come
\begin{equation}
\label{blur_multi}
z_i(x)=D[y](x) = B_i[y_i](x) + \eta(x), \qquad x \in X.
\end{equation}

\subsection{Spostamento della camera}

\subsection{Occlusione}
\section{Tampering detection}
