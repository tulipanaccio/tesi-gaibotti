\chapter{Introduzione}
\label{Introduzione}
\thispagestyle{empty}

\begin{quotation}
	{\footnotesize
		\noindent\emph{``Terence: Mi fai un gelato anche a me? Lo vorrei di pistacchio. \\
			Bud: Non ce l'ho il pistacchio. C'ho la vaniglia, cioccolato, fragola, limone e caff\`e. \\
			Terence: Ah bene. Allora fammi un cono di vaniglia e di pistacchio. \\
			Bud: No, non ce l'ho il pistacchio. C'ho la vaniglia, cioccolato, fragola, limone e caff\`e. \\
			Terence: Ah, va bene. Allora vediamo un po', fammelo al cioccolato, tutto coperto di pistacchio. \\
			Bud: Ehi, macch\'e sei sordo? Ti ho detto che il pistacchio non ce l'ho! \\
			Terence: Ok ok, non c'\`e bisogno che t'arrabbi, no? Insomma, di che ce l'hai? \\
			Bud: Ce l'ho di vaniglia, cioccolato, fragola, limone e caff\`e! \\
			Terence: Ah, ho capito. Allora fammene uno misto: mettici la fragola, il cioccolato, la vaniglia, il limone e il caff\`e. Charlie, mi raccomando il pistacchio, eh.''}
		\begin{flushright}
			Pari e dispari
		\end{flushright}
	}
\end{quotation}
\vspace{0.5cm}
Punti da sviluppare:

\section{Aumento nell'interesse dei contenuti multimediali, tra cui il \textit{monitoraggio video} (IoT, domotica,...)}
\section{Esempi videosorveglianza e webcam meteo}
\section{Problema dell'affidabilit\`a del contenuto video}
\begin{itemize}
	\item Vitale nel caso della videosorveglianza
	\item Caso webcam limitare traffico in rete evitando di inviare frame compromessi
	\item Qualit\`a del servizio
\end{itemize}
\section{Tampering Detection}
\section{Riassunto soa}
Enfatizzare il fatto che il lavoro presente in letteratura \`e legato solamente ad applicazioni di videosorveglianza
\section{Scopo della tesi}
\section{Soluzione proposta}
\section{Esperimenti fatti}
\section{Problemi rimasti aperti}
\section{Struttura della tesi}
La tesi \`e strutturata nel seguente modo.\\
Nel capitolo \ref{StatoArte} si mostra lo stato dell'arte.\\
Nel capitolo \ref{FormulazioneProblema} si illustra come \`e stato formalizzato il problema.\\
Nel capitolo \ref{SoluzioneProposta} si illustra la soluzione proposta per risolvere il problema.\\
Nel capitolo \ref{ProveSperimentali} si mostrano le prove realizzate per validare la soluzione proposta, descrivendo anche la realizzazione dei dataset e i risultati ottenuti.\\
Nel capitolo \ref{Conclusioni} si mostrano le prospettive future di ricerca e si tirano le conclusioni.
	
