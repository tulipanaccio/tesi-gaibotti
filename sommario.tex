\newpage
\chapter*{Sommario}

\addcontentsline{toc}{chapter}{Sommario}

Nell'ambito della \textit{video sorveglianza}, o pi\`u in generale nei sistemi video per il monitoraggio di ambienti, pu\`o capitare che eventi esterni (pioggia, vento, manomissione della camera) possano compromettere la qualit\`a delle immagini. La classe di algoritmi utilizzata per identificare, in maniera automatica,  questo tipo di eventi prende il nome di \textit{tampering detection}. La maggior parte di questi algoritmi richiede generalmente risorse di calcolo elevate e solitamente sono implementati in camere di video sorveglianza di fascia alta.

\vspace{0.5cm}
\noindent
Lo scopo della tesi \`e lo sviluppo di un algoritmo di tampering detection che possa essere utilizzato in un contesto \textit{embedded} e \textit{low-power}. La tecnica realizzata consiste nell'estrarre, da ciascun frame acquisito dalla camera, dei descrittori \textit{scalari}, la cui evoluzione viene monitorata nel tempo. L'evento di tampering viene identificato quando avviene un cambiamento nella distribuzione di questi indicatori. Per rendere l'analisi robusta si \`e introdotta una \textit{segmentazione} in varie aree della scena ripresa, creata durante una fase di \textit{configurazione} del sistema, in modo da estrarre gli indicatori solamente nelle zone in cui essi possiedono un andamento pi\`u o meno costante nel tempo.

%Negli ultimi anni si \`e potuto assistere a un massiccio incremento di dispositivi per acquisizioni di immagini e video, fenomeno legato a una societ\`a sempre pi\`u informatizzata. In questo contesto si inseriscono le camere di \textit{video monitoring}, le quali vengono utilizzate negli ambiti pi\`u disparati, dalla videosorveglianza in senso stretto al servizio di \textit{webcam} fornito dai principali siti web per il meteo. In questo tipo di applicazioni la camera \`e, solitamente, fissata a un supporto, in modo da monitorare una particolare scena. Durante la ripresa pu\`o accadere che eventi esterni (pioggia, vento, manomissione della camera \dots) possano compromettere la qualit\`a dell'acquisizione delle immagini (sfocature, occlusioni, spostamenti della camera) e, quindi, l'utilizzo effettivo dell'applicazione.\\
%Lo scopo della tesi \`e lo sviluppo di un algoritmo in grado di individuare, in maniera automatica, questo tipo di eventi. L'attivit\`a svolta si inserisce nel contesto dei sistemi \textit{embedded} e \textit{low-power}, dove diventano fondamentali la \textit{bassa complessit\`a computazionale} e il \textit{risparmio di memoria}. L'approccio utilizzato \`e stato quello di considerare dei particolari descrittori per ciascun frame, monitorati nel tempo, in modo da identificare un \textit{cambiamento} nella loro distribuzione, coincidente con un evento di \textit{tampering}. Una \textit{segmentazione} della scena ripresa dalla camera, calcolata durante una fase di \textit{configurazione} della stessa, permette di considerare solo le aree in cui questi indicatori hanno un comportamento \textit{stazionario} (magari!).
