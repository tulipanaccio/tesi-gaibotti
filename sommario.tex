\newpage
\chapter*{Sommario}

\addcontentsline{toc}{chapter}{Sommario}

Uno dei principali problemi, quando si ha a che fare con applicazioni di monitoraggio video, \`e quello di identificare quegli eventi che possono compromettere la corretta ripresa della scena da parte del sensore.
Pu\`o capitare, ad esempio, che dell'acqua piovana si depositi sulla lente della camera, rendendo l'immagine acquisita sfocata, oppure che qualcuno sposti la camera in modo che essa non riprenda pi\`u la scena che stava monitorando.
Il problema di individuare, in maniera automatica, questo tipo di eventi prende il nome di \textit{tampering detection}. 
Nella letteratura scientifica lo studio di questo problema si \`e concentrato solamente sulle applicazioni di \textit{videosorveglianza}, dove \`e necessario che la camera operi con una frequenza di acquisizione elevata.
In questo contesto immagini acquisite in istanti di tempo consecutivi hanno un alto grado di correlazione tra loro, in quanto il contenuto visivo varia molto poco.\\
Lo scopo della tesi \`e lo sviluppo di un algoritmo di tampering detection per sistemi di monitoraggio  adatto a operare con frequenze di acquisizioni basse, ad esempio un'immagine ogni minuto.
In questi casi, se consideriamo, ad esempio, la ripresa di una strada, in cui passano delle macchine o dei pedoni, abbiamo un'elevata dinamicit\`a che non permette di fare un confronto tra frame consecutivi per identificare gli eventi di nostro interesse. In aggiunta, abbiamo dei cambiamenti di luminosit\`a, tra un'immagine e la successiva, pi\`u sostanziali rispetto al caso di acquisizione continua.  
La nostra proposta \`e quella di monitorare nel tempo degli indicatori semplici, calcolati considerando solamente il \textit{contenuto visivo} delle singole immagini, dove una \textit{variazione sostanziale} \`e associata a un evento di tampering. 
Data l'alta variabilit\`a di questi indicatori abbiamo introdotto una \textit{segmentazione} della scena ripresa, estratta durante una fase di \textit{configurazione} dell'algoritmo, in modo da considerare solo le regioni in cui il monitoraggio risulta pi\`u efficace, in modo da . 
La tesi \`e stata svolta durante uno stage presso \textit{ST Microelectronics}, particolarmente interessata a sviluppare algoritmi intelligenti di processamento immagini da integrare nei propri dispositivi hardware, e a scenari di impiego per questi.
Prove sperimentali hanno confermato l'efficacia di utilizzare la segmentazione rispetto a considerare l'intera scena per individuare eventi di spostamento della camera. 