\newpage
\chapter*{Sommario}

\addcontentsline{toc}{chapter}{Sommario}

Nel campo dei sistemi di monitoraggio video, uno dei principali problemi \`e quello di identificare eventi che possano compromettere la corretta ripresa della scena.
Pu\`o capitare, ad esempio, che dell'acqua piovana si depositi sulla lente della camera, rendendo l'immagine acquisita sfocata, oppure che la camera si sposti, a causa di un intenzionale intervento umano o per eventi naturali quali una raffica di vento, e non riprenda pi\`u la scena da sorvegliare.\\
Il problema di individuare, in maniera automatica, questo tipo di eventi prende il nome di \textit{tampering detection}. 
Nella letteratura scientifica questo problema \`e stato affrontato solamente per applicazioni di \textit{videosorveglianza}, dove la camera opera con acquisizione continua, dispone di una certa potenza computazionale e viene alimentata a corrente.
Lo scopo della tesi \`e lo sviluppo di un algoritmo di tampering detection per sistemi \textit{embedded} e a basso consumo da utilizzarsi in scenari di monitoraggio. In particolare, l'algoritmo \`e caratterizzato da un basso carico computazionale ed \`e pensato per scenari, tipo il monitoraggio ambientale, dove il sistema, per ridurre il consumo energetico, acquisisce e analizza poche immagini al minuto o all'ora (\textit{framerate bassi}).
In questi casi scene ad \textit{alta dinamicit\`a}, come una strada in cui passano macchine e pedoni, non permettono di identificare eventi di tampering tramite un confronto tra frame consecutivi. 
Inoltre, operando a bassi framerate, si verificano sostanziali variazioni di luminosit\`a tra immagini consecutive. \\ 
L'algoritmo proposto si basa su indicatori, estratti dalle singole immagini, calcolati a bassa complessit\`a computazionale; tali indicatori vengono monitorati nel tempo attraverso tecniche \textit{sequenziali} e di \textit{outlier detection} per identificare l'istante in cui avviene l'evento di tampering.
Data l'alta variabilit\`a degli indicatori utilizzati, abbiamo introdotto una fase di \textit{segmentazione} della scena, in modo da limitare l'analisi ad alcune regioni specifiche:
questo permette di migliorare le prestazioni dell'algoritmo e diminuire il numero di falsi allarmi.\\
La tesi \`e stata svolta durante un lavoro di stage presso il gruppo \textit{Advanced System Technology} di \textit{STMicroelectronics}, dove la ricerca \`e volta allo sviluppo di algoritmi intelligenti di elaborazione immagini da integrare nei propri dispositivi embedded.
Sono stati messi a punto diversi sistemi di acquisizione operanti a diversi framerate, che hanno permesso di generare i dataset per testare l'efficacia della soluzione proposta e, in particolare, dei vantaggi nell'utilizzo della segmentazione a supporto del tampering detection. 
