\newpage
\chapter*{Sommario}

\addcontentsline{toc}{chapter}{Sommario}

Uno dei principali problemi, quando si ha a che fare con applicazioni di monitoraggio video, \`e quello di mantenere alta la qualit\`a delle immagini acquisite dal sensore.
Questo aspetto diventa pi\`u rilevante quando le camere utilizzate devono operare in ambienti esterni o pericolosi, dove fattori ambientali (pioggia, vento, riflessi causati dai raggi del sole \dots) o tentativi di \textit{manomissione} (spostamento della camera, occlusione dell'obiettivo, cambio della messa a fuoco dell'immagine \dots) possono compromettere la qualit\`a dei frame acquisiti, rendendoli quindi inutilizzabili per lo scopo dell'applicazione.
Il problema di individuare, in maniera automatica, questo tipo di eventi prende il nome di \textit{tampering detection}. 
Nella letteratura scientifica lo studio di questo problema si \`e concentrato solamente sulle applicazioni di \textit{videosorveglianza}, dove \`e necessario che la camera acquisisca a un \textit{framerate} elevato.
In questo contesto immagini acquisite in istanti di tempo consecutivi hanno un alto grado di correlazione tra loro, in quanto il contenuto visivo varia molto poco.\\
Lo scopo della testi \`e lo sviluppo di un algoritmo di tampering detection adatto a operare in condizioni di framerate \textit{basso}, dove i cambiamenti di luminosit\`a tra un'acquisizione e quella successiva sono pi\`u elevati, e le immagini, quindi, sono poco correlate tra di loro. 
La nostra proposta \`e quella di monitorare nel tempo degli indicatori semplici, calcolati considerando solamente il \textit{contenuto visivo} delle singole immagini, dove una \textit{variazione sostanziale} \`e associata a un evento di tampering. 
Data l'alta variabilit\`a di questi indicatori abbiamo introdotto una \textit{segmentazione} della scena ripresa, estratta durante una fase di \textit{configurazione} dell'algoritmo, in modo da considerare solo le regioni in cui il monitoraggio risulta pi\`u efficace. 
Le prove sperimentali, fatte durante uno stage presso \textit{ST Microelectronics}, hanno confermato l'efficacia di utilizzare la segmentazione rispetto a considerare l'intera scena per individuare eventi di spostamento della camera. 