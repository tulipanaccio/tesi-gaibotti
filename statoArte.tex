\chapter{Stato dell'arte}
\label{StatoArte}
\thispagestyle{empty}

%\begin{quotation}
%{\footnotesize
%\noindent{\emph{``Terence: Rotta a nord con circospezione \\
%Bud: Ehi, gli ordini li do io qui!\\
%Terence: Ok, comante\\
%Bud: Rotta a nord\\
%Terence: Soltanto?\\
%Bud: Con circospezione!''}
%}
%\begin{flushright}
%Chi Trova un Amico Trova un Tesoro
%\end{flushright}
%}
%\end{quotation}
\vspace{0.5cm}

\noindent In questo capitolo elenchiamo quelle che sono le principali tecniche, presenti nella letteratura scientifica, utilizzate per identificare tentativi di manomissione su camere di videosorveglianza. 
\section{Tampering Detection}
Nei moderni sistemi di videosorveglianza troviamo spesso algoritmi utilizzati per identificare particolari eventi all'interno della scena ripresa dalla camera. 
Ad esempio \`e possibile avere un software in grado di identificare le targhe delle automobili che superano il limite di velocit\`a , oppure la presenza di oggetti incustoditi in una stazione \cite{Targhe}.
Affinch\'e questi algoritmi funzionino correttamente, \`e importante che le immagini, che verranno poi processati da questi sistemi, mantengano una certa qualit\`a.

\subsection{Identificazione di occlusioni}
\subsection{Identificazione di spostamenti della camera}
\subsection{Identificazione di sfocature} 