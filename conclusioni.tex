\chapter{Conclusioni e direzioni future di ricerca}
\label{Conclusioni}
\thispagestyle{empty}

\noindent In questo lavoro di tesi abbiamo presentato una soluzione al problema, nei sistemi di monitoraggio video, di identificare particolari eventi che possano compromettere la corretta ripresa della scena, a cui si fa riferimento con il termine tampering.
In particolare abbiamo considerato l'utilizzo di camere intelligenti alimentate a batteria e lo scenario in cui la frequenza di acquisizione delle immagini da parte del sensore sia bassa, ad esempio un frame ogni minuto.\\
Abbiamo visto come il fatto di considerare dei framerate bassi complichi l'identificazione di eventi di tampering, in quanto le differenze tra frame consecutivi sono molto pi\`u elevate rispetto al caso di acquisizione continua. 
Inoltre, il fatto di avere algoritmi a bassa complessit\`a computazionale non ci permette di utilizzare degli indicatori troppo onerosi da estrarre dalle immagini.\\
La soluzione che abbiamo proposto \`e stata di utilizzare degli indicatori semplici, da estrarre in specifiche regioni della scena inquadrata dalla camera, e di monitorarli nel tempo con tecniche one-shot e sequenziali, in modo da individuare cambiamenti nel loro comportamento associabili a eventi di tampering.
Le regioni vengono estratte da un algoritmo di segmentazione della scena, che viene eseguito prima della messa in opera del sistema.\\
Abbiamo visto, infine, come le prove sperimentali condotte abbiano dimostrato che l'utilizzo della segmentazione della scena porti a dei miglioramenti nelle prestazioni dell'algoritmo, diminuendo il numero di falsi allarmi, senza aggiungere complessit\`a computazionale.
Un monitoraggio sequenziale sulla varianza dell'energia media del gradiente, inoltre, ha permesso di diminuire ulteriormente i falsi allarmi nell'identificazione degli eventi di sfocatura.\\
Gli sviluppi futuri riguarderanno il miglioramento dell'algoritmo e la sua integrazione su un dispositivo.
Per quanto riguarda l'identificazione di sfocature la ricerca dovr\`a essere condotta su come combinare il monitoraggio sequenziale con quello one-shot.
Per come \`e stato realizzato l'algoritmo al momento, infatti, le due tecniche sono eseguite in cascata, e il primo che identifica un cambiamento solleva l'allarme.
Dato che CONTINUARE!!!\\
Per quanto riguarda l'identificazione di spostamenti della camera un ulteriore sviluppo potrebbe riguardare l'integrazione delle informazioni estratte dalle immagini della camera con i dati estratti da dispositivi MEMS (\textit{Micro Electro-Mechanical Systems}), come \textit{sensori inerziali} o \textit{giroscopi}, in modo da poter validare gli allarmi ricevuti dalla camera e, quindi, ridurre ulteriormente i falsi allarmi.


